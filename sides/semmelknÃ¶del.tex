\begin{recipe}{Semmelknödel}%
    \index{Semmelknödel}%
    \index{Knödel!Semmel-}%
    \region{Österreich}%
    \source*{oesterreichische-kueche-nach-rokitansky}%
    \yield{4 Portionen}%

    \begin{ingredients}
        \entry[4]{Semmeln}
        \entry{Zwiebel}
        \entry{Petersiliengrün}
        \entry[50\gram]{Butter}
        \entry[\twosixteenth--\threesixteenth\liter]{Milch}
        \entry[1]{Ei}
        \entry{Salz}
        \entry{Muskat}
        \entry[2\el]{Mehl}
    \end{ingredients}

    \begin{instructions}
        Die Semmeln werden in kleine Würfel geschnitten, in 50\gram Butter, feingeschnittener Zwiebel und gehackter Petersilie geröstet, bis sie trocken sind.
        Milch wird mit Ei, Salz und Muskat abgesprudelt und über die in eine Schüssel geleerten Semmelwürfel gegossen.
        Das Ganze rasten lassen, bis die Masse angezogen hat.
        Danach mit 2 Esslöffel Mehl gut vermischen und anziehen lassen, mit nassen Händen unter festem Zusammenpressen 4 Knödel formen und in kochendem Salzwasser 15--20 Minuten kochen.
    \end{instructions}
\end{recipe}

\begin{recipe}{Pitabrot}
    \index{Pitabrot}%
    \index{Brot}%
    \source*{gu:oriental_basics}%
    \region{Naher Osten}

    \begin{ingredients}
        \entry[500\gram]{Mehl}
        \entry[30\gram]{Hefe}
        \entry[2\tl]{Zucker}
        \entry[2\el]{neutrales Öl}
        \entry[1]{Ei}
        \entry[4\el]{Sesamsamen}
        \entry[1\el]{Schwarzkümmelsamen}
        \entry{Salz}
    \end{ingredients}

    \begin{instructions}
        Das Mehl mit einem gehäuften \tl Salz in einer Schüssel mischen.
        Hefe mit 1\tl Zucker und 350\milliliter lauwarmes Wasser verühren.
        Hefewasse und 2\el Öl zum Mehl und gut duchkneten (mit Mixer).

        Teig mit Tuch abdecken un 40\minutes ruchen lassen.
        Teig mit den Händen mindestens 5\minutes kräftig durchkneten.
        Nochmal 40\minutes aufgehen lassen.

        Roht auf 250 Grad vorheizen (Umluft: 220 Grad).
        Teig durchkneten und halbieren.
        Jede Hälfte zu einem Fladen mit ca.\ 25\cm Durchmesser formen, jeweils auf ein Backblech legen.

        Ei mit dem restlichen Zucker und Öl verrühren.
        Fingerspitzen mit Öl anfeuchten und auf jeden Fladen ein rautenförmiges Muster drücken.
        Die Fladen mit der Eimischung einpinseln und mit Sesam und Schwarzkümmel bestreuen.

        Die Fladen nacheinander im Ofen (Mitte) 12--15\minutes backen.
        Nach der Hälfte der Zeit ein bisschen kaltes Wasser auf den Ofenboden schütten und die Tür gleich wieder zumachen.
    \end{instructions}
\end{recipe}

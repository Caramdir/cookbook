\begin{recipe}[\vegetarian]{Rösti}
    \label{rosti}
    \index{Rösti}%
    \region{Schweiz}%
    \ingredient{Erdäpfel}%
    \source{\nobreak\href{http://www.schweizer-kochrezepte.ch/rezepte/schweiz/zuercher_geschnetzeltes.html}{\url{www.schweizer-kochrezepte.ch}}}%

    \begin{ingredients}
        \entry[800\gram]{Erdäpfel}%
        \entry[80\gram]{Butter}%
        \entry[2\el]{Erdnussöl}%
        \entry[1\tl]{Salz}%
    \end{ingredients}

    \begin{instructions}
        Für die Rösti -- wenn immer möglich -- die Kartoffeln schon am Vortag kochen (nicht zu weich) und ungeschält kühl stellen.

        In einer grossen Teflonpfanne die Butter und das Erdnussöl erhitzen (das Erdnussöl kann auch durch Schweineschmalz ersetzt werden).
        Die Kartoffeln mit dem Salz mischen und hineingeben.
        Die Rösti locker verteilen, bei nicht zu hoher Temperatur ca.\ 10 Minuten die eine Seite braten. Die Rösti wenden und die andere Seite ebenfalls ca.\ 10 Minuten braten.
        Die Rösti müssen knusprig goldgelb gebacken sein.

        Die Hitze nun auf ein Minimum reduzieren (Stufe 1) und die Rösti warm halten.
    \end{instructions}
\end{recipe}

\begin{recipe}[\vegetarian]{Palatschinken}
    \index{Palatschinken}
    \region{Österreich}
    \source*{oesterreichische-kueche-nach-rokitansky}
    \yield{4 Portionen}

    \begin{ingredients}
        \entry[200\gram]{Mehl}
        \entry[2]{Eier}
        \entry[\threeeighth\liter]{Milch}
        \entry[1]{Dotter}
        \entry[1\prise]{Salz}
        \entry[100\gram]{Butter}
        \entry{Fett zum Backen}
        \entry{Marmelade}
        \entry{Staubzucker}
    \end{ingredients}

    \begin{instructions}
        In einer Schüssel Mehl, Milch, Eier, Eidotter und Salz zu einem glatten Teig verquirlen.
        Eventuell noch etwas Milch zugeben, so dass ein dünnflüssiger Teig entsteht. 
        Zugedeckt eine halbe Stunde ruhen lassen.

        In einer Pfanne ein wenig Fett erhitzen und so viel Teig eingießen, dass der Boden der Pfanne dünn bedeckt ist; dabei die Pfanne bewegen, so dass der Teig gleichmäßig verteilt wird.
        Bei mäßiger Hitze auf beiden Seiten hellbraun backen.
        Die fertigen Palatschinken auf einen warmen Teller geben.

        Sobald der Teig aufgebraucht ist, werden die Palatschinken mit verdünnter Marmelade bestrichen, zusammengerolld und vor dem Auftragen leicht überzuckert.
    \end{instructions}
\end{recipe}

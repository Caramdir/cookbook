\begin{recipe}[\vegetarian]{Kaiserschmarren}
\index{Kaiserschmarren}%
\index{Schmarren!Kaiser-}%
\region{Österreich}%
\source*{oesterreichische-kueche-nach-rokitansky}%
\yield{4--6 Portionen}%

\begin{ingredients}
    \entry[4]{Dotter}
    \entry[150\gram]{Mehl}
    \entry[ca.~\quarter\litre]{Milch}
    \entry{Salz}
    \entry[4]{Eiklar}
    \entry[30\gram]{Zucker}
    \entry[70\gram]{Butter}
    \entry[50\gram]{Rosinen}
    \entry{Staubzucker}
\end{ingredients}

\begin{instructions}
    Die Dotter werden mit Milch, Mehl und einer Prise Salz zu einem glatten Teig gerührt, der mit Zucker steif geschlagene Eiklarschnee daruntergezogen.
    In einer großen Omelettenpfanne lässt man 50\gram Butter aufschäumen, gießt die Teigmasse ein, streut Rosinen darüber und bäckt den Schmarren auf einer Seite goldbraun.
    Wendet un bäckt ihn auf der zweiten Seite, bis er innen fast durch ist, und zerreist den Schmarren mit zwei Gabeln in nicht zu kleine Stücke.
    20\gram frische Butter zugeben, den Schmarren bei starker Hitze kurz durchschwingen.
    Der goldbraun gebackene Kaiserschmarren wird angerichtet, mit Staubzucker bestreut und mit Zwetschkenröster serviert.
\end{instructions}
\end{recipe}

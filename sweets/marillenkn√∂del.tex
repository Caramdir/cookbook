\begin{recipe}[\vegetarian]{Marillenknödel}
\index{Marillenknödel}%
\index{Knödel!Marillen-}%
\region{Österreich}%
\ingredient{Marillen}%
\source*{oesterreichische-kueche-nach-rokitansky}%

\begin{ingredients}
    \entry{Erdäpfelteig oder Topfenteig}
    \entry[10--15]{Marillen}
    \entry{Würfelzucker}
    \entry[80\gram]{Butter}
    \entry[80\gram]{Semmelbrösel}
    \entry{Staubzucker}
\end{ingredients}

\begin{instructions}
    Der gewählte Teig wird zu einer dicken Rolle geformt und in Scheiben geschnitten.
    Von den Marillen wird mit einem Kochlöffelstiel der Kern herausgestochen und durch einen halben Würfelzucker ersetzt.
    Jede Marille wird in ein Stück Teig eingeschlagen und gut verschlossen geformt in leicht kochendem Salzwasser, bei nicht ganz geschlossenem Deckel, kochen.
    Aus dem Wasser heben und abtropfen.
    Butter und geriebene Semmelbrösel rösten und die Knödel darin rollen.
    Angezuckert servieren.
\end{instructions}
\end{recipe}

\begin{recipe}[\vegetarian]{Topfenschmarren}
\index{Topfenschmarren}%
\index{Schmarren!Topfen-}%
\region{Österreich}%
\ingredient{Topfen}%
\source*{oesterreichische-kueche-nach-rokitansky}%
\yield{4 Portionen}%

\begin{ingredients}
    \entry[250\gram]{trockener Topfen}
    \entry[4]{Eidotter}
    \entry[\threeeighth\liter]{Sauerrahm}
    \entry[\prise]{Salz}
    \entry[80\gram]{Mehl}
    \entry[20\gram]{Zucker}
    \entry[4]{Eiklar}
    \entry[60\gram]{Butter}
    \entry{Staubzucker}
\end{ingredients}

\begin{instructions}
    Zerriebener Topfen wird nach und nach mit 4 Eidottern, Salz und Sauerrahm verrührt.
    Mehl einrühren und zum Schluss Zucker und steifgeschlagenene Eischnee darunterziehen.
    In einer Pfanne Butter erhitzen und die Masse eingießen.
    Im Rohr lichtbraun backen, mit einer Backschaufel zerstechen und kurz nachbacken.
    Der Schmarren soll schön saftig bleiben und wird beim Auftragen angezuckert.
    Dazu kann man eine Fruchsoße reichen.
\end{instructions}
\end{recipe}

\begin{recipe}[\vegetarian]{Kaiserschmarren}
\index{Kaiserschmarren}%
\index{Schmarren!Kaiser-}%
\region{Österreich}%
\source*{wikibook-de}%

\begin{ingredients}
    \entry[300\gram]{Mehl}
    \entry[\half\litre]{Milch}
    \entry[6]{Eier}
    \entry[50\gram]{zerlassene Butter}
    \entry[80\gram]{Rosinen}
    \entry[2\el]{Rum}
    \entry{Salz}
    \entry{Fett zum Backen}
    \entry{Staubzucker zum Bestreuen}
    \entry{Kompott oder Apfelmus}
\end{ingredients}

\begin{instructions}
    Die Eier trennen und das Eiweiß zu Schnee schlagen.
    Teig aus Mehl, Milch und Eigelb herstellen.
    Die zerlassene Butter unterrühren, dann den Rum.
    Den Eischnee unterziehen.
    Leicht salzen.

    Fett in Pfanne erhitzen.
    Den Teig ca.\ 1\cm hoch eingießen und anbacken lassen.
    Die Rosinen einstreuen, den Teig umdrehen und anbacken lassen.
    Den Teig in etwas größere Stücke zerstoßen und goldgelb backen lassen.
    Nach Belieben etwas zuckern und kurz rösten (karamellisieren).

    Den Schmarrn auf Teller anrichten und mit Staubzucker bestreuen.
    Mit Kompott, Preiselbeeren, Heidelbeeren, Apfelmus oder Eis servieren.
\end{instructions}
\end{recipe}

\begin{recipe}{Grießschmarren}
\index{Grießschmarren}%
\index{Schmarren!Grieß-}%
\region{Österreich}%
\ingredient{Grieß}%
\source*{oesterreichische-kueche-nach-rokitansky}%

\begin{ingredients}
    \entry[\half\liter]{Milk}
    \entry[60\gram]{Butter}
    \entry{Salz}
    \entry[200\gram]{grober Grieß}
    \entry[50\gram]{Rosinen}
    \entry[1]{Eidotter}
    \entry[50\gram]{Butter zum Backen}
    \entry{Staubzucker}
\end{ingredients}

\begin{instructions}
    Die Milch wird mit Butter und einer Prise Salz aufgekocht, den groben Grieß einlaufen lassen und langsam dick verkochen, vom Feuer nehmen, Rosinen und 1 Eidotter einrühren.
    In einer Pfanne läßt man die Butter heiß werden, gibt die Grießmasse hinein und läßt sie im Rohr unter öfteren Zerkleinern goldbraun und körnig backen.

    Angerichtet mit Staubzucker bestreuen, mit Zwetschkenröster oder Fruchsaft servieren.
\end{instructions}
\end{recipe}

\begin{recipe}{Hummus}
    \index{Hummus}
    \ingredient{Kichererbsen}%
    \ingredient{Tahin}%
    \region{Naher Osten}%
    \source*{gu:oriental_basics}

    \begin{ingredients}
        \entry[400\gram]{Kichererbsen aus der Dose}
        \entry[2]{Knoblauchzehen}
        \entry[2\el]{Olivenöl}
        \entry[3\el]{Zitronensaft}
        \entry[3\el]{Tahin}
        \entry{Salz}
        \entry[\half\tl]{gemahlener Kreuzkümmel}
        \entry[1\el]{Butter}
        \entry[1\tl]{edelsüßes Paprikapulver}
    \end{ingredients}

    \begin{instructions}
        Die Kichererbsen im Sieb gründlich kalt abbrausen und abtropfen lassen.
        Knoblauch schälen und würfeln.

        Kicherebsen mit Knoblauch, Olivenöl, Zitronensaft, 2--3\el Wasser und Tahin zu einer cremigen Paste pürieren.
        Mit Salz und Kreuzkümmel abschmecken und in eine kleine Schüssel füllen.
        In der Mitte mit dem Löffel eine kleine Mulde eindrücken.

        Die Butter in einem Pfännchen nur zerlaufen lassen, braun soll sie nicht werden, dann das Paprikapulver unterrühren.
        Die rote Butter in die Mulde und über den Rest vom Hummus laufen lassen.
    \end{instructions}
\end{recipe}

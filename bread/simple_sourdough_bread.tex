\begin{recipe}[\vegan]{Simple Sourdough Bread}
    %with Quinoa & Vegetables
    \index{simple sourdough bread}
    \index{sourdough bread}
    \source{\href{https://alexandracooks.com/2017/10/24/artisan-sourdough-made-simple-sourdough-bread-demystified-a-beginners-guide-to-sourdough-baking/}{Alexandra's Kitchen}}

    \begin{ingredients}
        \entry[75\gram]{bubbly, active starter}
        \entry[375\gram]{warm water}
        \entry[500\gram]{bread flour}
        \entry[10\gram]{salt}
    \end{ingredients}

    \begin{instructions}
        If you live in a warm, humid environment, 50\gram starter should suffice. If you plan on doing an overnight rise, 50\gram also should suffice. If you want  to speed things up or if you live in a cold environment, consider using 100\gram starter. Note: If you use 100\gram of starter, your dough may rise more quickly, so keep an eye on it. As always, rely on the visual cues (doubling in volume) when determining when the bulk fermentation is done. 

        \emph{Make the dough:}
        In the evening, whisk the starter and water together in a large bowl with a fork or spatula.
        Add the flour and salt.
        Mix to combine, finishing by hand if necessary to form a rough dough.
        Cover with a damp towel and let rest for 30 minutes. 

        \emph{Stretch and fold:} After 30 minutes, grab a corner of the dough and pull it up and into the center.
        Repeat until you’ve performed this series of folds 4 to 5 times with the dough.
        Let dough rest for another 30 minutes and repeat the stretching and folding action.
        If you have the time: do this twice more for a total of 4 times in 2 hours.

        \emph{Bulk Fermentation (first rise):}
        Cover the bowl with a towel and let rise overnight at room temperature, about 8 to 10 hours at 70°F (21°C) or even less if you live in a warm environment.
        The dough is ready when it has doubled in size, has a few bubbles on the surface, and jiggles when you move the bowl from side to side.
        (Note: If you are using 100\gram of starter, this may take less than 8 to 10 hours. If you live in a warm, humid environment, too, this may take even less time (4 to 5 hours… in the late spring/early summer my kitchen is 78ºF and the bulk fermentation takes 6 hours).
        It is best to rely on visual cues (doubling in volume) as opposed to time to determine when the bulk fermentation is done.
        A straight-sided vessel makes monitoring the bulk fermentation especially easy because it allows you to see when your dough has truly doubled.)

        \emph{Shape:}
        In the morning, coax the dough onto a lightly floured surface.
        Gently shape it into a round: fold the top down to the center, turn the dough, fold the top down to the center, turn the dough; repeat until you’ve come full circle.
        If you have a bench scraper, use it to push and pull the dough to create tension.

        Let the dough rest seam side up rest for 30 minutes.
        Meanwhile, line an 8\inch (20\cm) bowl or proofing basket with a towel (flour sack towels are ideal) and dust with flour (preferably rice flour, which doesn’t burn the way all-purpose flour does).
        Using a bench scraper or your hands, shape it again as described in step 4.
        Place the round into your lined bowl, seam side up.

        \emph{Proof (second rise):} Cover the dough and refrigerate for 1 hour or for as long as 48 hours. (Note: I prefer to let this dough proof for at least 24 hours prior to baking.)  

        Place a Dutch oven in your oven, and preheat your oven to 550\degF (290\degC).
        Cut a piece of parchment to fit the size of your baking pot.

        \emph{Score:}
        Place the parchment over the dough and invert the bowl to release.
        Using the tip of a small knife or a razor blade, score the dough however you wish — a simple “X” is nice.
        Use the parchment to carefully transfer the dough into the preheated baking pot.

        \emph{Bake:}
        Carefully cover the pot, close the oven, and reduce the heat to 450\degF (230\degC).
        Bake the dough for 30 minutes, covered.
        Remove the lid, lower the temperature to 400\degF (200\degC) and continue to bake for 10 – 15 minutes more.
        If necessary, lift the loaf out of the pot, and bake directly on the oven rack for the last 5 to 10 minutes.
        Cool on a wire rack for 1 hour before slicing.
    \end{instructions}
\end{recipe}

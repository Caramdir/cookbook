\begin{recipe}{Pizzateig}
    \index{Pizzateig}%
    \index{Teig}%
    \region{Italien}%

    \begin{ingredients}
        \entry[25g]{fresh yeast; or 1\tbsp dry yeast; or 2\tbsp instant dry yeast}
        \entry[\half\tsp]{sugar}
        \entry[250\ml]{warm water}
        \entry[500\gram]{flour (Italian type 0)}
        \entry[1\tsp]{fine sea salt}
        \entry[1\tbsp]{olive oil}
    \end{ingredients}

    \begin{instructions}
        This is the dough for a typical Neapolitan pizza: soft and easy to chew with crispy base or \emph{cornicione.}

        In a medium bowl, mix the fresh yeast with sugar and whisk in the warm water.
        Leave 10 minutes until foamy.
        For other yeasts, follow the manufacturer's instructions.

        Sift flour and salt into a large bowl and make a well in the center.
        Pour in the mixture of yeast and then olive oil.
        Stir well with a spatula; then, manually, knead until the dough is well blended and smooth.
        Place on a lightly floured surface, wash and dry hands; then knead vigorously for 5--10 minutes until the dough has a smooth, shiny and elastic texture (five minutes in the case of having your hands warm; 10 minutes if the hands are cold!).
        No need to add more flour at this step: the more humid it is, the better.
        If you notice that the dough is sticky, flour your hands, not the deough.
        The dough should be soft.
        If it \emph{really} is very soft, knead with a little more flour.

        To check if the dough is ready, you can roll it in the form of thick sausage, take one end in each hand, lift the dough and stretch out, moving gently; it should stretch quite easily.
        Place in a greased bowl, cover with plastic wrap or a damp cloth and let rise in a warm place until it reaches twice its size (about an hour).

        Uncover the dough, give it a tap to remove air and then arrange it on a lightly floured work surface.
        Split into 2 parts and form 2 balls.
        Place the balls on separate nonstick paper, cover them with plastic wrap loosely and let rise for 60--90 minutes.
        You can use them whenever you want.
    \end{instructions}
\end{recipe}

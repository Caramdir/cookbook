\begin{recipe}[\vegan]{George's Dal}
    \index{George's Dal}
    \index{Dal!Georges' Dal}
    \region{Sri Lanka}
    \ingredient{Linsen}
    \source{\href{https://www.bbc.co.uk/food/recipes/georges_dal_75678}{BBC Food}}
    \yield{4 servings}

    \begin{ingredients}
        \entry[200\gram]{red lentils}
        \entry[\half\tsp]{turmeric}
        \entry[2]{garlic cloves, peeled and kept whole}
        \entry[250\ml]{coconut milk}
        \entry[25\gram]{butter}
        \entry[1\tbsp]{oil}
        \entry[10]{fresh curry leaves}
        \entry[1\tsp]{black mustard seeds}
        \entry{pinch of dried red chilli, thinly sliced}
        \entry[1]{small red onion, thinly sliced}
        \entry{salt, to taste}
    \end{ingredients}

    \begin{instructions}
        Wash the lentils several times in clean water to remove the starch.

        Place the lentils in a large saucepan and add 500\ml water, the turmeric and garlic. Bring to the boil, then reduce the heat and cook for around 15 minutes. Stir in the coconut milk and continue to cook for a further 5 minutes or until the lentils are cooked and the dal has thickened. Season to taste with salt.

        In a separate small frying pan, melt the butter with the oil on a low heat. Add the curry leaves, mustard seeds and chilli. Gently fry for 1–2 minutes, then add the onion and continue to cook for a few minutes, stirring all the time until the onions are golden and caramelised.

        Add this mixture to the lentils and stir together. Remove the garlic cloves before serving. If you want to make the dal look attractive, garnish with a few fried curry leaves and sliced chilli (optional).
    \end{instructions}
\end{recipe}

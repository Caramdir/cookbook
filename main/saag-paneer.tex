\begin{recipe}{Saag Paneer}
    \index{Saag Paneer}
    \region{Indien}
    \ingredient{Spinat}
    \index{Paneer}
    \source{\href{https://www.bbcgoodfood.com/recipes/saag-paneer}{BBC goodfood}}
    \yield{6 servings}

    \begin{ingredients}
        \entry[2\tbsp]{ghee}
        \entry[1\tsp]{turmeric}
        \entry[1\tsp]{Kashmiri chilli powder}
        \entry[450\gram]{paneer, cut into 3\cm cubes}
        \entry[500\gram]{spinach, mature fresh or frozen}
        \entry[1]{large onion, finely chopped}
        \entry[3]{garlic cloves}
        \entry{thumb-sized piece of ginger}
        \entry[1]{green chilli, roughly chopped, (include seeds for extra spice)}
        \entry[1\tsp]{garam masala}
        \entry[\half]{lemon, juiced, to serve}
    \end{ingredients}

    \begin{instructions}
        Melt the ghee, whisk in with the turmeric and chilli powder, then add the cubed paneer and toss well.
        Set aside.
        If using frozen spinach, microwave for 3-5 mins, then place in a sieve and squeeze out most of the water.
        If using fresh spinach, place in a colander, pour over boiling water, drain and cool, then put in a tea towel and squeeze out most of the water.
        Roughly chop.

        Blitz the onion with the garlic, ginger and green chilli.
        Cook the paneer in a large non-stick frying pan over medium heat for around 8 mins, tossing the pan so they become golden all over.
        Remove and set aside on a plate, leaving spices behind in the pan.
        Tip the onion mix into the pan, add a pinch of salt and turn the heat down.
        Fry until caramel coloured, around 10 mins, adding a splash of water if it looks a little dry.
        Add the garam masala, stir to coat the onion mix, fry for 2 mins.

        Add the spinach and cook for a further 2-3 mins, adding 100\ml water to release all the flavours from the bottom of the pan.
        Add the paneer and cook for 2-3 mins to heat through.
        Spoon into bowls and squeeze over a little lemon juice, to serve.
    \end{instructions}
\end{recipe}

\begin{recipe}{Zürcher Geschnetzeltes}
    \index{Zürcher Geschnetzeltes}%
    \region{Schweiz}%
    \ingredient{Kalb}%
    \ingredient{Champignon}%
    \source{\nobreak\href{http://www.schweizer-kochrezepte.ch/rezepte/schweiz/zuercher_geschnetzeltes.html}{\url{www.schweizer-kochrezepte.ch}}}%
    \cf{rosti}

    \begin{ingredients}
        \entry[560\gram]{Kalbschnitzelfleisch (geschnetzelt)}%
        \entry[2\el]{Erdnussöl}
        \entry[160\gram]{Champignons}%
        \entry[4\el]{geschälte und gehackte Zwiebel}%
        \entry[1\el]{Mehl}
        \entry[3\el]{Butter}
        \entry[200\milliliter]{Weißwein}%
        \entry[400\milliliter]{Schlagobers}%
        \entry{Pfeffer}
    \end{ingredients}

    \begin{instructions}
        Das Geschnetzelte mit Salz und Pfeffer aus der Mühle würzen und in zwei Teile teilen.
        Eine grosse Teflonpfanne mit der Hälfte des Erdnussöles sehr gut erhitzen, einen Teil des Fleisches beigeben und kurz anbraten.
        Das Fleisch aus der Pfanne nehmen, warm stellen.
        Mit dem restlichen Fleisch ebenso verfahren.

        In derselben Pfanne die Butter erhitzen, die Zwiebeln beigeben und dünsten.
        Die in Scheiben geschnittenen Champignons dazugeben, mit dem Mehl stäuben und vermischen.
        Weisswein beigeben und zur Hälfte einreduzieren.

        Den Bratensaft vom Fleisch und das Schlagobers dazugeben und alles zur gewünschten Konsistenz einkochen.

        Mit Salz und Pfeffer abschmecken.

        Das Fleisch in die Sauce geben (nicht mehr kochen) und mischen.
    \end{instructions}
\end{recipe}

\begin{recipe}{Fasolakia}
    \index{Fasolakia}
    \region{Griechenland}
    \source{http://mamastaverna.com/fasolakia-green-beans/}
    \ingredient{Fisolen}

    \begin{ingredients}
        \entry[2\pound]{green beans}
        \entry[2\pound]{tomatoes}
        \entry[1\pound]{potatoes}
        \entry[\half\cup]{olive oil}
        \entry[1\pound]{onions}
        \entry{garlic}
        \entry[1\bunch]{parsley}
        \entry{pepper}
        \entry{salt}
    \end{ingredients}

    \begin{instructions}
        Slice or chop a few cloves of garlic.
        Grate, chop or purée the tomatoes.
        Wash and chop the parsley.
        Snap the ends off the green beans and discard them.
        Rinse and drain the bean middles.
        Cut the stem ends off the onions, then halve them lengthwise.
        Peel and slice them.
        Peel the potatoes and slice them rather thickly.

        Cover the bottom of the pot with olive oil, about \half cup.
        Heat the oil on high, add the onions and saute them for about 5 minutes.
        Add the garlic, 6 peppercorns, and half the parsley, and saute for another 5 minutes. 
        Then add the tomatoes, 1\half\tsp salt, and 1\tsp pepper, and bring to a boil.

        After boiling the tomatoes for 5 minutes or so, add the potatoes and stir them into the sauce.
        Put the green beans on top.
        Do not stir them in!
        Keep them as a layer on top of the potatoes!
        Make sure the liquid comes up past the potato layer to approximately the middle of the green bean layer.
        If not, add water to make it do so.
        It's okay if the liquid even comes up to the top of the green bean layer, better to have the dish a little soupy than to have it not cook up properly, or even burn.
        Sprinkle with remaining parsley.

        Cover and simmer vigorously for approximately 1 hour.
        Note: If you're using a nonstick pot, you can simmer pretty vigorously.
        If you're using a good heavy-bottomed pot that's not nonstick, you can simmer pretty vigorously but you'd better check every 15 minutes or so to make sure the dish isn't sticking to the bottom.
        If it is, you need to stir it a bit and maybe turn the heat down some.
        If you have to keep the heat lower to keep the dish from sticking, you might need to plan on an extra hour or so of cooking.
        And if you're using a crappy thin-bottomed pot because that's all you have, you can still make a great pot of beans, but make it a day ahead of time so that you can cook it slowly for as long as necessary.

        You are not aiming for crisp, bright-green beans with this recipe.
        The beans should be soft and sweet.
        Be sure to soak up the delicious sauce with some bread.
        This dish is just as good the next day.
    \end{instructions}
\end{recipe}

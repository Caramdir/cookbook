\begin{recipe}{Fazzoletti (Pasta Handkerchiefs)}
    \label{fazzoletti}
    \index{fazzoletti}%
    \index{pasta}
    \region{Italien}
    \source{New York Times \cite{nyt-fazzoletti-chunky-pesto, nyt-minimalist-pasta-in-the-rought}}
    \cf{chunky_pesto}%
    \yield{4 servings}

    \begin{ingredients}
        \entry[2\cups]{all-purpose flour, plus more as needed}
        \entry[1\tsp]{salt, plus more as needed}
        \entry[2]{whole eggs}
        \entry[3]{egg yolks}
    \end{ingredients}

    \begin{instructions}
        Pulse flour and salt in a food processor once or twice.
        Add the eggs and yolks, and turn the machine on.
        Process just until a ball begins to form, about 30 seconds.
        Add a few drops of water if the dough is dry and grainy; add a tablespoon of flour if the dough sticks to the side of the bowl.
        Turn the dough out of the food processor, sprinkle it with a little flour, cover it with plastic or a cloth, and let it rest for about 30 minutes.
        (At this point, you may refrigerate the dough, wrapped in plastic, until you’re ready to roll it out, for up to 24 hours.)

        Bring a large pot of water to a boil and salt it.
        Divide the dough in half.
        Turn one half of the dough onto a lightly floured surface and roll it into a large rectangle no thicker than \quarter\inch and ideally closer to \eighth\inch, adding additional flour sparingly as necessary.
        Repeat with the rest of the dough.

        Cut into squares no larger than 4 inches across.
        Drop the squares into the water and cook until tender, 2 to 3 minutes.
    \end{instructions}
\end{recipe}

\begin{recipe}{Topfenteig zum Füllen}%
    \index{Topfenteig zum Füllen}%
    \index{Knödel!Topfenteig}%
    \region{Österreich}%
    \ingredient{Topfen}%
    \source*{oesterreichische-kueche-nach-rokitansky}%
    \yield{etwa 10 Knödel}%

    \begin{ingredients}
        \entry[250\gram]{10\%iger Topfen}
        \entry[1]{Ei}
        \entry[60\gram]{Butter}
        \entry[125\gram]{Mehl}
        \entry[2\el]{Grieß}
    \end{ingredients}

    \begin{instructions}
        Passierter Topfen wird nach und nach mit flüssiger Butter, dem Ei, einer Prise Salz, Mehl und Grieß zu einem Teig vermischt.
        20--30 Minuten kalt stellen.
        Sodann ein Probeknöderl anfertigen, welches man in Salzwasser 7 bis 8 Minuten kocht.
        Ist der Teig zu weich, fügt man noch etwas Mehl bei, ist er zu fest, muss man noch flüssige Butter einrühren.
    \end{instructions}
\end{recipe}

\begin{recipe}[\vegetarian]{Erdäpfelteig für Knödel}%
    \index{Erdäpfelteig}%
    \index{Knödel!Erdäpfelteig}%
    \region{Österreich}%
    \ingredient{Erdäpfel}%
    \source*{oesterreichische-kueche-nach-rokitansky}%
    \yield{etwa 10 Knödel}%

    \begin{ingredients}
        \entry[400\gram]{mehlige Erdäpfel}
        \entry[125\gram]{Mehl}
        \entry[25\gram]{Butter}
        \entry[1]{Eidotter}
        \entry{Salz}
    \end{ingredients}

    \begin{instructions}
        Die am Vortag in der Schale gekochten Erdäpfel werden geschält, fein gerissen oder zerdrückt, mit Mehl, flüssiger Butter, Eidotter und Salz zu einem nicht zu festen Teig geknetet.
        Nur kurz abarbeiten, de der Teig sonst in der Konsistenz nachläßt.
        Vor dem Weiterverarbeiten enen Probeknödel in kochendes Salzwasser legen.
        Ist er zu weich, dem Teig noch etwas Mehl oder Grieß zufügen.
        Ist der Teig zu fest, etwas Butter einkneten.
    \end{instructions}
\end{recipe}
